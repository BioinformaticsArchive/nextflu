\documentclass[11pt]{article}
%%%%%%%%%%%%%%%%%%%%%%%%%%%%%%%%%%%%%%%%%%%%%%%%%%%%%%%%%%%%%%%%%%%%%%%%%
%%%%%%%%%%%%%%%%%%%%%%%%%%%%%%%%%%%%%%%%%%%%%%%%%%%%%%%%%%%%%%%%%%%%%%%%%
\usepackage[english]{babel}
\usepackage[utf8x]{inputenc}
\usepackage{amsmath,amsfonts,amssymb,eucal,eurosym,textcomp}
\usepackage{color}
\usepackage{graphicx}
\usepackage[numbers]{natbib}
%\usepackage{lineno}
%\linenumbers

\title{Estimation of frequency trajectories}

\begin{document}
\maketitle
What we got is a series of dated observations that either belong to a particular clade or not
\begin{equation}
    (t_i, c_i) \; \mathrm{for}\; i\in 1,\ldots,n \; \mathrm{and}\; c_i\in[0,1],\; t_i = date
\end{equation}
the frequency of the clade in question is a continuous time trajectory $x(t)$. Given the 
trajectory, the likelihood of observing the data is
\begin{equation}
    P(\{(t_i, c_i)\}|x(t)) \sim \prod_{i=1}^n x(t_i)^{c_i}(1-x(t_i))^{1-c_i}
\end{equation}
We want to estimate $x(t)$, which changes over time due to selection and chance. This could 
modeled as 
\begin{equation}
    \dot{x} = x(1-x) (s-\bar{s}) + \sqrt{2Dx(1-x)}\eta + \mu(2x-1)
\end{equation}
where $\eta$ is a white noise Wiener process and $D$ is the diffusion constant (what other 
people would call the inverse effective population size) and the last term could be 
the mutational influx. The probability of a particular trajectory 
\begin{equation}
    \log P(x(t) | \{(t_i, c_i)\} ) \sim \sum_{i=1}^n c_i\log(x(t_i)) + (1-c_i)\log(1-x(t_i))
     - \int dt \frac{(\dot{x}-x(1-x)(s-\bar{s}))^2}{4Dx(1-x)}
\end{equation}
One conveniently discretizes the time integral to 
\begin{equation}
    \sum_j \Delta t \frac{(x_{j+1}-x_j-x_j(1-x_j)(s-\bar{s}))^2}{4Dx_j(1-x_j)} 
\end{equation}
while interpreting the stochastic ODE as Ito.
An alternative variable would be $z = x/(1-x)$ (such that $x = z/(1+z)$ and 
$1-x = \frac{1}{1+z}$). Ignoring the determinant for now, the posterior becomes
\begin{equation}
\begin{split}
    \log P(x(t) | \{(t_i, c_i)\} ) & \sim \sum_{i=1}^n c_i\log(z(t_i)/(1+z(t_i))) + (1-c_i)\log(1/(1+z(t_i)))\\
    &-\sum_j \Delta t (1+z_j)^2\frac{(z_{j+1}-z_j(1+s-\bar{s}))^2}{4Dz_j}     \\
    &=\sum_{i=1}^n c_i\log(z(t_i)) - \log(1+z(t_i))\\
    &-\sum_j \Delta t (1+z_j)^2\frac{(z_{j+1}-z_j(1+s-\bar{s}))^2}{4Dz_j} 
\end{split}
\end{equation}
The mix of exponential growth and diffusion means that this isn't very simple, but it 
is not too bad after all. Taking derivatives with respect to $z_k$, we find
\begin{equation}
\begin{split}
    &\frac{\partial}{\partial z_k}\log P(x(t) | \{(t_i, c_i)\} )  = 0 = 
            \sum_{i=1 | t_i \in \Delta t_k}^n \left[\frac{c_i}{z_k} - \frac{1}{1+z_k}\right] \\
           &-\Delta t (1+z_k)\frac{(z_{k+1}-z_k(1+s-\bar{s}))^2}{2Dz_k}
            +\Delta t (1+z_k)^2\frac{(z_{k+1}-z_k(1+s-\bar{s}))^2}{4Dz_k^2}\\
           &-\Delta t (1+z_k)^2\frac{(z_{k+1}-z_k(1+s-\bar{s}))(1+s-\bar{s})}{2Dz_k}  
           -\Delta t (1+z_{k-1})^2\frac{(z_{k}-z_{k-1}(1+s-\bar{s}))}{2Dz_{k-1}}
\end{split}
\end{equation}
We need to solve this for all $z_k$ simultaneously to obtain the most likely trajectory. 
Regorganizing slightly and ignoring higher order terms, we have
\begin{equation}
\begin{split}
     0  \approx  &4D \sum_{i=1 | t_i \in \Delta t_k}^n \left[\frac{c_i}{z_k} - \frac{1}{1+z_k}\right] \\
           &-\Delta t \left[2z_k(1+z_k)+(1+z_k)^2\right](z_{k+1}/z_k-(1+s-\bar{s}))^2\\
           &-\Delta t 2(1+z_k)^2 (z_{k+1}/z_k+z_{k}/z_{k-1}-2(1+s-\bar{s}))  
\end{split}
\end{equation}
More promising, however, might be straight numerical minimization of the log-likelihood. 
We could in principle fit $s$ for individual clades while calculating $\bar{s}$ as an average
\begin{equation}
    \bar{s}(t_i) = \sum_j x_j(t_i)s_j
\end{equation}
where $x_j$ is the frequency of clade $j$. What is not obvious to me at this point is how to
consist the fitness of the subclades with that of the parent clade. 

\end{document}
